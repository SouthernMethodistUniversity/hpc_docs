\section{Message Passing Interface (MPI)}

\begin{frame}{Motivation}
\begin{itemize}
\item MPI is a specification of what the interface should look like and what it should do
\begin{itemize}
\item Separate processes on each node communicate by sending and receiving data over a network
\item MPI can be used for parallelism on a single node as well
\end{itemize}
\item An MPI implementation is a set of libraries that allow for multiple nodes to be used together via message passing
\item Many higher-level languages and libraries support or use MPI
\item MPI has become the industry standard for distributed-memory programming
\end{itemize}
\end{frame}

\begin{frame}{MPI Implementations}
\begin{itemize}
\item Open source
\begin{itemize}
\item MPICH
\item OpenMPI
\item MVAPICH
\end{itemize}
\item Closed source
\begin{itemize}
\item Intel MPI (based on MPICH)
\item Mellanox HPC-X MPI (based on OpenMPI)
\end{itemize}
\end{itemize}
\end{frame}

\begin{frame}{Compiling MPI Programs}
Depending on your programming language and the specific MPI implementation,
these wrapper scripts can have different names
\begin{itemize}
\item C++: mpicxx or mpic++
\item C: mpicc
\item Fortran 90/95/2003: mpif90
\item Fortran 77: mpif77
\end{itemize}
\end{frame}

\begin{frame}{Running MPI Programs}
\begin{itemize}
\item Running MPI batch jobs on ManeFrame is almost identical to running serial
      and OpenMP batch jobs. However, when running MPI jobs, we must tell the
      queueing system a few additional pieces of information:
\begin{enumerate}
\item How many total nodes we want to reserve on the machine?
\item How many total cores do we want to reserve on the machine?
\item How do you want to distribute tasks on each node?
\item How many MPI tasks do you actually want to run?
\end{enumerate}
\item We have two key ways to control execution of parallel batch jobs:
\begin{itemize}
\item Controlling how the job is reserved
\item Controlling how the MPI job is executed
\end{itemize}
\end{itemize}
\end{frame}

\subsection{Examples}
\subsubsection{Estimating \(\pi\)}

\begin{frame}{Estimating \(\pi\)}
\begin{listing}[H]
\inputminted[firstline=1, lastline=7]{C}{examples/mpi/mpi_monte_carlo_pi.c}
\caption{Includes and function prototypes}
\end{listing}
\end{frame}

\begin{frame}{Estimating \(\pi\)}
\begin{listing}[H]
\inputminted[firstline=35, lastline=48]{C}{examples/mpi/mpi_monte_carlo_pi.c}
\caption{Function definitions.}
\end{listing}
\end{frame}

\begin{frame}{Estimating \(\pi\)}
\begin{listing}[H]
\inputminted[firstline=9, lastline=15]{C}{examples/mpi/mpi_monte_carlo_pi.c}
\caption{Argument check and variable initialization.}
\end{listing}
\end{frame}

\begin{frame}{Estimating \(\pi\)}
\begin{listing}[H]
\inputminted[firstline=16, lastline=21]{C}{examples/mpi/mpi_monte_carlo_pi.c}
\caption{Initialize MPI, get the global number of tasks, and get the process rank.}
\end{listing}
\end{frame}

\begin{frame}{Estimating \(\pi\)}
\begin{listing}[H]
\inputminted[firstline=22, lastline=27]{C}{examples/mpi/mpi_monte_carlo_pi.c}
\caption{Get number of hits per rank via summation reduction.}
\end{listing}
\end{frame}

\begin{frame}{Estimating \(\pi\)}
\begin{listing}[H]
\inputminted[firstline=28, lastline=33]{C}{examples/mpi/mpi_monte_carlo_pi.c}
\caption{If rank zero, report estimation of \(\pi\). All ranks finalize and exit.}
\end{listing}
\end{frame}

\begin{frame}{Estimating \(\pi\)}
\begin{listing}[H]
\inputminted[firstline=6, lastline=8]{Bash}{examples/mpi/README.md}
\caption{Build executable.}
\end{listing}
\end{frame}

\begin{frame}{Estimating \(\pi\)}
\begin{listing}[H]
\inputminted[firstline=1, lastline=8]{Bash}{examples/mpi/mpi_monte_carlo_pi.sbatch}
\caption{Reguest compute resources.}
\end{listing}
\end{frame}

\begin{frame}{Estimating \(\pi\)}
\begin{listing}[H]
\inputminted[firstline=10]{Bash}{examples/mpi/mpi_monte_carlo_pi.sbatch}
\caption{Setup environment and run.}
\end{listing}
\end{frame}

