\section{Version Control Systems}

\begin{frame}{Overview}
\begin{itemize}
\item Version Control (aka \emph{revision control} or \emph{source control})
lets you track the history of your files over time. Why do you care? So
when you mess up you can easily get back to a previous version that
worked.\footnote{Adapted from
\href{http://betterexplained.com/articles/a-visual-guide-to-version-control/}{A
visual guide to version control}.}
\item You've probably invented your own simple version control system in the
past without realizing it. Do you have an directories with files like
this?
\end{itemize}
\end{frame}

\begin{frame}{Ad Hoc Version Control}
\begin{itemize}
\item \mintinline{sh}{my_function.c}
\item \mintinline{sh}{my_function2.c}
\item \mintinline{sh}{my_function3.c}
\item \mintinline{sh}{my_function4.c}
\item \mintinline{sh}{my_function_old.c}
\item \mintinline{sh}{my_function_older.c}
\item \mintinline{sh}{my_function_even_older.c}
\end{itemize}
\end{frame}


\begin{frame}{Ad Hoc Version Control}
It's why we use "Save As"; you want to save the new file without writing
over the old one. It's a common problem, and solutions are usually like
this:


\begin{itemize}
\item
  Make a single backup copy, e.g. \mintinline{sh}{Document.old.txt}.
\item
  If we're clever, we add a version number or date, e.g.
  \mintinline{sh}{Document_V1.txt} or \mintinline{sh}{DocumentMarch2012.txt}.
\item
  We may even use a shared folder so other people can see and
  edit files without sending them by email. Hopefully they rename the
  file after they save it.
\end{itemize}
\end{frame}




\begin{frame}{Why use a VCS?}
\begin{itemize}
\item Our shared folder/naming system is fine for class projects or one-time
papers, but is exceptionally bad for software projects. Do you imagine
that the Windows source code sits in a shared folder named something
like "Windows7-Latest-New", for anyone to edit? Or that every programmer
just works on different files in the same folder?

\item For projects that are large, fast-changing, or have multiple authors, a
Version Control System (VCS) is critical. Think of a VCS as a "file
database", that helps to track changes and avoid general chaos.
\end{itemize}
\end{frame}

\begin{frame}{VCS Features}
\begin{description}
\item[Backup and Restore] Files are saved as they are edited, and
  you can jump to any moment in time. Need that file as it was on March
  8? No problem.
\item[Synchronization] Allows people to share files and stay up
  to date with the latest version.
\item[Short-term undo] Did you try to "fix" a file and just mess
  it up? Throw away your changes and go back to the last "correct"
  version in the database.
\item[Long-term undo] Sometimes we mess up bad. Suppose you made
  a change a year ago, and it had a bug that you never caught until now.
  Jump back to the old version, and see what change was made that day.
  Maybe you can fix that one bug and not have to undo your work for the
  whole year?
\end{description}
\end{frame}

\begin{frame}{VCS Features}
\begin{description}
\item[Track Changes] As files are updated, you can leave messages
  explaining why the change happened (these are stored in the VCS, not
  the file). This makes it easy to see how a file is evolving over time,
  and why it was changed.
\item[Track Ownership] A VCS tags every change with the name of
  the person who made it, which can be hepful for laying blame \emph{or}
  giving credit.
\item[Sandboxing] Plan to make
  a big change? You can make temporary changes in an isolated area, test
  and work out the kinks before "checking in" your set of changes.
\item[Branching and merging] A larger sandbox. You can branch a
  copy of your code into a separate area and modify it in isolation
  (tracking changes separately). Later, you can merge your work back
  into the common area.
\end{description}
\end{frame}

\begin{frame}{Shared Folders}
Shared folders are quick and simple, but can't provide these critical
features.
\end{frame}

\begin{frame}{General definitions}
Most version control systems involve the following concepts, though the
labels may be different.

Basic setup:

\begin{description}
\item[Repository (repo)] The database storing the files.
\item[Server] The computer storing the repo.
\item[Client] The computer connecting to the repo.
\item[Working Copy] Your local directory of files,
  where you make changes.
\item[Main] The primary location for code in the repo.
  Think of code as a family tree, where ``main'' is the trunk.
\end{description}
\end{frame}

\begin{frame}{Basic Actions}
\begin{description}
\item[Add] Put a file into the repo for the first time, i.e.
  begin tracking it with Version Control.
\item[Revision] What version a file is on (v1, v2, v3, etc.).
\item[Head/Tip] The latest revision in the repo.
\item[Check out] Download a file from the repo.
\item[Check in] Upload a file to the repository (if it has
  changed). The file gets a new revision number, and people can "check
  out" the latest one.
\end{description}
\end{frame}
\begin{frame}{Basic Actions}
\begin{description}
\item[Checkin Message] A short message describing what was
  changed.
\item[Changelog/History] A list of changes made to a file since
  it was created.
\item[Update/Sync] Synchronize your files with the latest from
  the repository. This lets you grab the latest revisions of all files.
\item[Revert] Throw away your local changes and reload the latest
  version from the repository.
\end{description}
\end{frame}

\begin{frame}{More Advanced Actions}
\begin{description}
\item[Branch] Create a separate copy of a file/folder for private
  use (bug fixing, testing, etc). Branch is both a verb ("branch the
  code") and a noun ("Which branch is it in?").
\item[Diff/Change/Delta] Finding the differences between two
  files. Useful for seeing what changed between revisions.
\item[Merge/Patch] Apply the changes from one file to another, to
  bring it up-to-date. For example, you can merge features from one
  branch into another.
\end{description}
\end{frame}
\begin{frame}{More Advanced Actions}
\begin{description}
\item[Conflict] When pending changes to a file contradict each
  other (both changes cannot be applied automatically).
\item[Resolve] Fixing the changes that contradict each other and
  checking in the final version.
\item[Locking] Taking control of a file so nobody else can edit
  it until you unlock it. Some version control systems use this to avoid
  conflicts.
\item[Breaking the lock] Forcibly unlocking a file so you can
  edit it. It may be needed if someone locks a file and goes on
  vacation.
\item[Check out for edit] Checking out an "editable" version of a
  file. Some VCSes have editable files by default, others require an
  explicit command.
\end{description}
\end{frame}

\begin{frame}{Typical Scenario}

\begin{itemize}
\item
  Alice adds a file (ShoppingList.txt) to the repository.
\item
  Alice checks out the file, makes a change (puts "milk" on the list),
  and checks it back in with a checkin message ("Added delicious
  beverage.").
\item
  The next morning, Bob updates his local working set and sees the
  latest revision of ShoppingList.txt, which contains "milk".
\item
  Bob adds "donuts" to the list, while Alice also adds "eggs" to the
  list.
\item
  Bob checks the list in, with a checking message "Mmmmm, donuts".
\item
  Alice updates her copy of the list before checking it in, and notices
  that there is a conflict. Realizing that the order of items doesn't
  matter, she merges the changes by putting both "donuts" and "eggs" on
  the list, and checks in the final version.
\end{itemize}
\end{frame}

\subsection{Git}

\begin{frame}{Git}
\begin{itemize}
\item Originally released in 2005 (by
\href{https://en.wikipedia.org/wiki/Linus_Torvalds}{Linus Torvalds}
himself!).
\item \href{https://en.wikipedia.org/wiki/Git_(software)}{Git} was 
one of the first version control systems that followed a
\emph{distributed revision control} model (DRCS), in which it is no
longer required to have a single server that all clients connect with.
\item DRCS follows a peer-to-peer approach in which each peer's
working copy of the codebase is a fully-functional repository. These
work by exchanging patches (sets of changes) between peers, resulting in
some
\href{https://en.wikipedia.org/wiki/Distributed_revision_control\#Distributed_vs._centralized}{key
benefits over previous centralized systems}.
\end{itemize}
\end{frame}


\begin{frame}{Git Commands}
\begin{description}
\item[Clone] Primary mechanism for retrieving
  a local copy of a Git repository.
\item[Pull] Fetches and merges changes on the remote
  server to your working repository.
\item[Push] Opposite of \mintinline{sh}{pull}, this sends all
  changes in your local repository to a remote repository.
\end{description}
\end{frame}

\begin{frame}{Git Resources}

\begin{itemize}
\item
  \href{http://git-scm.com/}{Main Git site}
\item
  \href{http://www.atlassian.com/git/tutorial}{Git tutorials}
\item
  \href{http://git-scm.com/book}{Git book chapters}
\end{itemize}
\end{frame}

